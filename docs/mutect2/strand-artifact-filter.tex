\documentclass[a4paper]{article}

%% Language and font encodings
\usepackage[english]{babel}
\usepackage[utf8x]{inputenc}
\usepackage[T1]{fontenc}

%% Sets page size and margins
\usepackage[a4paper,top=3cm,bottom=2cm,left=3cm,right=3cm,marginparwidth=1.75cm]{geometry}

%% Useful packages
\usepackage{amsmath}
\usepackage{graphicx}
\usepackage[colorinlistoftodos]{todonotes}
\usepackage[colorlinks=true, allcolors=blue]{hyperref}

\title{Mutect2 Strand Artifact Filter}
\author{Takuto Sato}

\begin{document}
\maketitle

\section{Introduction}

(Describe the strand artifact)
Assumptions: 
biallelic

\section{The Probabilistic Model}
\subsection{Graphical Model and Conditional Distributions}

(Describe that the model applies to a single site only. Extension over multiple sites is coming.)

\begin{figure}
\centering
\includegraphics[width=0.3\textwidth]{strand_artifact_pgm.png}
\caption{\label{fig:frog}The probabilistic graphical model of the new strand artifact filter.}
\end{figure}

\begin{itemize}
	\item $z \sim \text{Multinomial}(\pi)$, where $z  \in \{ \text{artifact on forward strand}, \text{artifact on reverse strand}, \text{no artifact} \}$, represents the presence of the strand artifact in either the forward or reverse reads. For instance, artifact on forward strand means that the artifactual evidence for the alternate allele was detected in the forward (+) reads but not in the reverse (-) reads. (REWORD)
	\item $f \sim \text{Unif}(0, 1)$ represents the true allele fraction at the site.
	\item $\epsilon \sim \text{Beta}(\alpha, \beta)$ represents the allele fraction of strand-specific artifact. (Not strictly true...?)
	\item $x^+ | f, \epsilon, z \sim$ A mixture of binomials, defined as:
		\begin{itemize}
			\item $x^+ | f, \epsilon, z = \text{Artifact+} \sim \text{Binomial} (n^+, f + \epsilon(1-f))$
			\item $x^+ | f, \epsilon, z = \text{Artifact-} \sim \text{Binomial} (n^+, f)$
			\item $x^+ | f, \epsilon, z = \text{NoArtifact} \sim \text{Binomial} (n^+, f)$
		\end{itemize}
		The conditional distributions of $x^-$ are similar.
\end{itemize}

Having observed the read counts in the + and - directions, we can compute the posterior probabilities of the latent variable $z$, as described below.

\subsection{Posterior Probabilities}

Below we derive the posterior probability of discrete latent variable $z$.

\begin{equation}
\begin{split}
P(z=Artifact+ |x^+, x^-) & \propto p(z) p(x^+, x^- | z=Artifact+) \\
		    & = p(z) \iint_{f, \epsilon}  p(x^+, x^-, f, \epsilon | z) \,df\,d\epsilon \\
		    & = p(z) \iint_{f, \epsilon}  p(f) p(\epsilon) p(x^+, x^- | z, f, \epsilon) \,df\,d\epsilon \\
                     & = p(z) \iint_{f, \epsilon}  p(f) p(\epsilon) p(x^+ | z, f, \epsilon) p(x^- | z, f, \epsilon) \,df\,d\epsilon \\
                     & = p(z) \iint_{f, \epsilon}  p(\epsilon) p(x^+ | z, f, \epsilon) p(x^- | z, f, \epsilon) \,df\,d\epsilon \\
                     & = p(z) \iint_{f, \epsilon}  Beta(\epsilon|\alpha, \beta) Binomial(x^+ | f + \epsilon(1-f), n^+) Binomial(x^- | f, n^-) \,df\,d\epsilon \\
\end{split}
\end{equation}

The posterior probability of no artifact follows the same derivation up to the second to last step. Here we get more simplifications because the conditional probabilities of $x^+$ and $x^-$ do not depend on $\epsilon$:

\begin{equation}
\begin{split}
P(z=NoArtifact |x^+, x^-) & \propto p(z) p(x^+, x^- | z=NoArtifact) \\
		    & = p(z) \iint_{f, \epsilon}  p(\epsilon) p(x^+ | z, f, \epsilon) p(x^- | z, f, \epsilon) \,df\,d\epsilon \\
		    & = p(z) \int_{f}  p(x^+ | z, f) p(x^- | z, f) \,df \int_{\epsilon}  p(\epsilon) d\epsilon \\
                     & = p(z) \int_{f}  Binomial(x^+ | f, n^+) Binomial(x^- | f, n^-) \,df \\
\end{split}
\end{equation}


Note that once two of the three are calculated we may simply subtract their sum from 1 to get the third. No simple analytical solution exists for the double integrals in (1) and thus we must integrate it numerically.

\subsection{Numerical Integration}

\section{Generative Model}

\section{Concerns}
What happens if we only get reads in one direction e.g. near the end of targets?


\begin{enumerate}
\item Draw a discrete latent variable $z$ from the prior weight $pi$. $z$ may take on three possible values. $Art+$, $Art-$
\end{enumerate}



\end{document}