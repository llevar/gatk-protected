\documentclass[a4paper]{article}

%% Language and font encodings
\usepackage[english]{babel}
\usepackage[utf8x]{inputenc}
\usepackage[T1]{fontenc}

%% Sets page size and margins
\usepackage[a4paper,top=3cm,bottom=2cm,left=3cm,right=3cm,marginparwidth=1.75cm]{geometry}

%% Useful packages
\usepackage{amsmath}
\usepackage{graphicx}
\usepackage[colorinlistoftodos]{todonotes}
\usepackage[colorlinks=true, allcolors=blue]{hyperref}

\title{Mutect2 Strand Artifact Filter}
\author{Takuto Sato}

\begin{document}
\maketitle

\section{Introduction}

(Describe the strand artifact)
Assumptions: 
biallelic

\section{The Probabilistic Model}
\subsection{Graphical Model and Conditional Distributions}

(Describe that the model applies to a single site only. Extension over multiple sites is coming.)

\begin{figure}
\centering
\includegraphics[width=0.3\textwidth]{strand_artifact_pgm.png}
\caption{\label{fig:frog}The probabilistic graphical model of the new strand artifact filter.}
\end{figure}

\begin{itemize}
	\item $z \sim \text{Multinomial}(\pi)$, where $z  \in \{ \text{artifact on forward strand}, \text{artifact on reverse strand}, \text{no artifact} \}$, represents the presence of the strand artifact in either the forward or reverse reads. For instance, artifact on forward strand means that the artifactual evidence for the alternate allele was detected in the forward (+) reads but not in the reverse (-) reads. (REWORD)
	\item $f \sim \text{Unif}(0, 1)$ represents the true allele fraction at the site.
	\item $\epsilon \sim \text{Beta}(\alpha, \beta)$ represents the allele fraction of strand-specific artifact. (Not strictly true...?)
	\item $x^+ | f, \epsilon, z \sim$ A mixture of binomials, defined as:
		\begin{itemize}
			\item $x^+ | f, \epsilon, z = \text{Artifact+} \sim \text{Binomial} (n^+, f + \epsilon(1-f))$
			\item $x^+ | f, \epsilon, z = \text{Artifact-} \sim \text{Binomial} (n^+, f)$
			\item $x^+ | f, \epsilon, z = \text{NoArtifact} \sim \text{Binomial} (n^+, f)$
		\end{itemize}
		The conditional distributions of $x^-$ are similar.
\end{itemize}

Having observed the read counts in the + and - directions, we can compute the posterior probabilities of the latent variable $z$, as described below.

\subsection{Posterior Probabilities}

Below we derive the posterior probability of discrete latent variable $z$.

\begin{equation}
\begin{split}
P(z=Artifact+ |x^+, x^-) & \propto p(z) p(x^+, x^- | z=Artifact+) \\
		    & = p(z) \iint_{f, \epsilon}  p(x^+, x^-, f, \epsilon | z) \,df\,d\epsilon \\
		    & = p(z) \iint_{f, \epsilon}  p(f) p(\epsilon) p(x^+, x^- | z, f, \epsilon) \,df\,d\epsilon \\
                     & = p(z) \iint_{f, \epsilon}  p(f) p(\epsilon) p(x^+ | z, f, \epsilon) p(x^- | z, f, \epsilon) \,df\,d\epsilon \\
                     & = p(z) \iint_{f, \epsilon}  p(\epsilon) p(x^+ | z, f, \epsilon) p(x^- | z, f, \epsilon) \,df\,d\epsilon \\
                     & = p(z) \iint_{f, \epsilon}  Beta(\epsilon|\alpha, \beta) Binomial(x^+ | f + \epsilon(1-f), n^+) Binomial(x^- | f, n^-) \,df\,d\epsilon \\
\end{split}
\end{equation}

The posterior proability of no artifact follows the same derivation up to the second to last step. Here we get more simplifications because the conditional probabilities of $x^+$ and $x^-$ do not depend on $\epsilon$:

\begin{equation}
\begin{split}
P(z=NoArtifact |x^+, x^-) & \propto p(z) p(x^+, x^- | z=NoArtifact) \\
		    & = p(z) \iint_{f, \epsilon}  p(\epsilon) p(x^+ | z, f, \epsilon) p(x^- | z, f, \epsilon) \,df\,d\epsilon \\
		    & = p(z) \int_{f}  p(x^+ | z, f) p(x^- | z, f) \,df \int_{\epsilon}  p(\epsilon) d\epsilon \\
                     & = p(z) \int_{f}  Binomial(x^+ | f, n^+) Binomial(x^- | f, n^-) \,df \\
\end{split}
\end{equation}


Note that once two of the three are calculated we may simply subtract their sum from 1 to get the third. No simple analytical solution exists for the double integrals in (1) and thus we must integrate it numerically.

\subsection{Generative Model}

\begin{enumerate}
\item Draw a discrete latent variable $z$ from the prior weight $pi$. $z$ may take on three possible values. $Art+$, $Art-$
\end{enumerate}





\subsection{How to add Comments}

Comments can be added to your project by clicking on the comment icon in the toolbar above. % * <john.hammersley@gmail.com> 2016-07-03T09:54:16.211Z:
%
% Here's an example comment!
%
To reply to a comment, simply click the reply button in the lower right corner of the comment, and you can close them when you're done.

Comments can also be added to the margins of the compiled PDF using the todo command\todo{Here's a comment in the margin!}, as shown in the example on the right. You can also add inline comments:

\todo[inline, color=green!40]{This is an inline comment.}

\subsection{How to add Tables}

Use the table and tabular commands for basic tables --- see Table~\ref{tab:widgets}, for example. 

\begin{table}
\centering
\begin{tabular}{l|r}
Item & Quantity \\\hline
Widgets & 42 \\
Gadgets & 13
\end{tabular}
\caption{\label{tab:widgets}An example table.}
\end{table}

\subsection{How to write Mathematics}

\LaTeX{} is great at typesetting mathematics. Let $X_1, X_2, \ldots, X_n$ be a sequence of independent and identically distributed random variables with $\text{E}[X_i] = \mu$ and $\text{Var}[X_i] = \sigma^2 < \infty$, and let
\[S_n = \frac{X_1 + X_2 + \cdots + X_n}{n}
      = \frac{1}{n}\sum_{i}^{n} X_i\]
denote their mean. Then as $n$ approaches infinity, the random variables $\sqrt{n}(S_n - \mu)$ converge in distribution to a normal $\mathcal{N}(0, \sigma^2)$.


\subsection{How to create Sections and Subsections}

Use section and subsections to organize your document. Simply use the section and subsection buttons in the toolbar to create them, and we'll handle all the formatting and numbering automatically.

\subsection{How to add Lists}

You can make lists with automatic numbering \dots

\begin{enumerate}
\item Like this,
\item and like this.
\end{enumerate}
\dots or bullet points \dots
\begin{itemize}
\item Like this,
\item and like this.
\end{itemize}

\subsection{How to add Citations and a References List}

You can upload a \verb|.bib| file containing your BibTeX entries, created with JabRef; or import your \href{https://www.overleaf.com/blog/184}{Mendeley}, CiteULike or Zotero library as a \verb|.bib| file. You can then cite entries from it, like this: \cite{greenwade93}. Just remember to specify a bibliography style, as well as the filename of the \verb|.bib|.

You can find a \href{https://www.overleaf.com/help/97-how-to-include-a-bibliography-using-bibtex}{video tutorial here} to learn more about BibTeX.

We hope you find Overleaf useful, and please let us know if you have any feedback using the help menu above --- or use the contact form at \url{https://www.overleaf.com/contact}!

\bibliographystyle{alpha}
\bibliography{sample}

\end{document}