\documentclass[a4paper]{article}

%% Language and font encodings
\usepackage[english]{babel}
\usepackage[utf8x]{inputenc}
\usepackage[T1]{fontenc}

%% Sets page since and margins
\usepackage[a4paper,top=3cm,bottom=2cm,left=3cm,right=3cm,marginparwidth=1.75cm]{geometry}

%% Useful packages
\usepackage{amsmath}
\usepackage{graphicx}
\usepackage[colorinlistoftodos]{todonotes}
\usepackage[colorlinks=true, allcolors=blue]{hyperref}

\title{Mutect2 Strand Artifact Filter}
\author{Takuto Sato}

\begin{document}
\maketitle

\section{Introduction}


Strand artifact filter captures false positives that arise from . This model is a build 

\section{The Probabilistic Model}
\begin{figure}
\centering
\includegraphics[width=0.3\textwidth]{strand_artifact_pgm.png}
\caption{\label{fig:frog}The probabilistic graphical model}
\end{figure}

\begin{itemize}
	\item $z \sim \text{Multinomial}(\pi)$ is a latent variable, encoded as a 3-dimensional vector of the probability of strand artifact in forward reads (index 0), reverse reads (1), or neither (2). For instance, $z = [0.9, 0.03, 0.07]$ means that we believe that any given \textit{forward} ref read is subject to strand artifact with probability 0.9.
	\item $f \sim \text{Unif}(0, 1)$ is the alt allele fraction at the locus
	\item $\epsilon \sim \text{Beta}(\alpha, \beta)$ is the probability that, given there is strand artifact, we mistakenly read an ref allele as alt
	\item $x^+ | f, \epsilon, z \sim$ is the number of alt reads in the forward reads. It's a mixture of binomials, defined as:
	$$
	x^+ | f, \epsilon, z \sim
		\begin{cases}
			\text{Binomial} (n^+, f + \epsilon(1-f)) & z = \mathrm{Art+}\\
			 \text{Binomial} (n^+, f) 			& z = \mathrm{Art-} \\
			 \text{Binomial} (n^+, f)			& z = \mathrm{noArt}
		\end{cases}
	$$
			%\item $x^+ | f, \epsilon, z = \text{Artifact+} \sim \text{Binomial} (n^+, f + \epsilon(1-f))$
			%\item $x^+ | f, \epsilon, z = \text{Artifact-} \sim \text{Binomial} (n^+, f)$
			%\item $x^+ | f, \epsilon, z = \text{NoArtifact} \sim \text{Binomial} (n^+, f)$
\end{itemize}



We compute the conditional distributions of $x^-$  analogously.

Having observed the read counts in the forward and reverse directions, we can compute the posterior probabilities of the latent variable $z$ as described in the next section.

\subsection{Posterior Probabilities}

Below we derive the unnormalized posterior probability of strand artifact in forward reads ($z = art+$), given that we observed $x^+$ forward and $x^-$ reverse alt reads at a given locus. We use a shorthand $z_0$ to denote $z=art+$ for conciseness. 

\begin{equation}
\begin{split}
P(z_0 |x^+, x^-) & \propto p(z_0) p(x^+, x^- | z_0) \\
& = p(z_0) \iint_{f, \epsilon}  p(x^+, x^-, f, \epsilon | z_0) \,df\,d\epsilon \\
& = p(z_0) \iint_{f, \epsilon}  p(f) p(\epsilon) p(x^+, x^- | z_0, f, \epsilon) \,df\,d\epsilon \\
& = p(z_0) \iint_{f, \epsilon}  p(f) p(\epsilon) p(x^+ | z_0, f, \epsilon) p(x^- | z_0, f, \epsilon) \,df\,d\epsilon \\
& = p(z_0) \iint_{f, \epsilon}  p(\epsilon) p(x^+ | z_0, f, \epsilon) p(x^- | z_0, f, \epsilon) \,df\,d\epsilon \\
& = p(z_0) \iint_{f, \epsilon}  \mathrm{Beta}(\epsilon|\alpha, \beta) \mathrm{Binomial}(x^+ | f + \epsilon(1-f), n^+) \mathrm{Binomial}(x^- | f, n^-) \,df\,d\epsilon \\
\end{split}
\end{equation}

The posterior probability of no artifact follows the same derivation up to the second to last step. It reduces to a single integral over $f$ because the conditional probabilities of $x^+$ and $x^-$ do not depend on $\epsilon$. We again use a shorthand $z_2$ for $z=\mathrm{noArt}$

\begin{equation}
\begin{split}
P(z_2 |x^+, x^-) & \propto p(z_2) p(x^+, x^- | z_2) \\
		    & = p(z_2) \iint_{f, \epsilon}  p(\epsilon) p(x^+ | z_2, f, \epsilon) p(x^- | z_2, f, \epsilon) \,df\,d\epsilon \\
		    & = p(z_2) \int_{f}  p(x^+ | z_2, f) p(x^- | z_2, f) \,df \int_{\epsilon}  p(\epsilon) d\epsilon \\
                     & = p(z_2) \int_{f}  \mathrm{Binomial}(x^+ | f, n^+) \mathrm{Binomial}(x^- | f, n^-) \,df \\
\end{split}
\end{equation}

\end{document}